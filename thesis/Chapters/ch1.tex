
\chapter{Chapter 1. Introduction}%Be sure to include Chapter 1. before you write the name of your chapter. Name all remaining chapters in the same manner.

Accurate characterization of the distribution of water content within the soil column by land surface models is critical for numerical weather prediction (NWP), operational decision making preceding and during drought and flood events, and for downstream datasets aiding assessment of vegetation health, crop yield prediction, and fire risk characterization \cite{koster_contribution_2010}\cite{otkin_assessing_2016}\cite{case_role_2023}. Noah-LSM has helped to address these needs by serving as the land surface component coupled to NWP models including the Weather Research and Forecasting Model (WRF), the Global Forecast System (GFS) \cite{jin_sensitivity_2010}\cite{mitchell_ncep_2005}, and climate models including the NCEP Climate Forecast System (CFSv2) \cite{saha_ncep_2014}. Noah-LSM also aids National Weather Service forecasts and US Drought Monitor designations within frameworks like the Short-Term Research, Prediction, and Transition high-resolution implementation of the Land Information System (SPoRT-LIS) \cite{case_nasa_2022}\cite{case_assessment_2014}, and supports research and derived product development by providing soil states for NLDAS datasets \cite{ek_implementation_2003}.

NLDAS has provided the community with reliable multi-model land surface states and associated forcings in a near real-time capacity since 1999 \cite{cosgrove_real-time_2003}, with phase 2 of the project also contributing a retrospective climatology extending back to 1979 \cite{xia_continental-scale_2012}.

Data-driven modeling techniques like neural networks introduce the ability to approximate the highly nonlinear and conditional relationships between predictor and target datasets. This flexibility is accomplished by learning a sequence of transformations which are encoded as a composition of alternating high-dimensional matrix operations and element-wise nonlinear functions, and which serve as a mapping from the vector of predictors to a corresponding target vector. In the context of physical modeling, this

\section{History of NLDAS and Noah-LSM}

Chapter titles should begin with the word chapter and the appropriate number followed by a period. After typing the chapter heading, then type the chapter title. This template automatically formats your chapter titles. Just do not forget to include the chapter heading when you type the chapter name.

All paragraphs throughout your thesis should begin with an ½ inch indentation. It should be double-spaced throughout. Since this is a formal document, do not use contractions. Remember that paragraphs should consist of at least two sentences. Figure 1.1 lists 11 common grammar mistakes. Please avoid these!

\begin{figure}[ht]
    \centering
    \includegraphics[width=\textwidth]{Figures/figure 1.1.jpg}
    \caption[11 Most Common Grammar Mistakes Employees Make: I'm purposely making this longer to extend to two lines.]{11 Most Common Grammar Mistakes Employees Make. When labeling your figures, single-space if captions extend to two lines}
    \label{fig 1.1}
\end{figure}

\section{Deep Learning for Time Series Modeling}

If your document includes many symbols or acronyms, you may include a List of Symbols, Abbreviations, \textit{etc}. If you want a symbol/abbreviation included in the List of Symbols, be sure to create an entry for it first on the List of Symbols Glossaries.tex file. Once it is created, then you can insert it with a glossaries command. For example, the current temperature outside is 100\glspl{deg}.

You can capitalize your symbols or make them plural by using different commands included with the glossaries package. However, only those symbols that are actually referenced in the body of your thesis will be present in the List of Symbols. Below are a few more symbol examples.

\gls{grav}

\gls{wf}

\gls{alp}

\gls{theta}

\gls{te}

\gls{q10}



