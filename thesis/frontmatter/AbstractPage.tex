% The top of your abstract will fill out automatically once you fill in the required fields on the main.tex file. In this file, you will provide your abstract body. Type your abstract body at the bottom of this page directly below the \doublespacing command.

\chapter{Abstract}
     \begin{center}
        \large
        \singlespacing
        \textbf{\thesistitle}\\
        \vspace{0.5cm}
        \large
        \textbf{\studentname}\\
        \vspace{0.5cm}
        \normalsize
        \ifdefined\thesis
        \textbf{A thesis submitted in partial fulfillment of the requirements \\for the degree of \degree}\\
        \else
        \ifdefined\dissertation
        \textbf{A dissertation submitted in partial fulfillment of the requirements \\for the degree of \degree}\\
        \else
        \textbf{Please identify this document as either a thesis or dissertation on the main.tex in the section at the top that must be filled out.}\\
    \fi
    \fi
        \vspace{1cm}
        \textbf{\department}

        \vspace{0.25cm}

        \ifdefined\jointuni
        \textbf{The University of Alabama in Huntsville and  \jointuni}
        \else
        \textbf{The University of Alabama in Huntsville}
    \fi


        \vspace{0.1cm}
        \textbf{\gradmonth\ \gradyear}



    \end{center}
\vspace{0.1cm}

%****************************************************
%Enter the body of your abstract below. Remember there is a 150 word limit!
%****************************************************
\doublespacing

This work examines the ability of deep learning time series generative models to accurately and efficiently emulate the hourly temporal dynamics of the Noah Land Surface Model (Noah-LSM) out to a 2 week forecast horizon, given atmospheric forcings and static parameterization provided by the second phase North American Land Data Assimilation System (NLDAS-2) framework. Results from multiple neural network architectures are compared alongside variations in prediction target, loss function characteristics, and model properties. The most performant model types are subsequently evaluated with respect to forecast distance, daily and annual seasonality, and against a variety of regional scenarios, including several extreme event case studies. Ultimately, we present a software system for developing and testing neural networks that use time-varying and static data to estimate temporal dynamics, with the goal of providing a foundation for similar data-driven modeling techniques to be implemented within the upcoming third phase of the NLDAS data record.

\clearpage

