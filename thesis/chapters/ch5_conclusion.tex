\chapter{Chapter 5. Conclusion and Future Work}

\section{Conclusion}

In this work, we have advanced the science of data-driven modeling of Earth system processes by showing that relatively small ANNs are capable of reasonably emulating the 1-dimensional dynamics of the Noah land surface model at a high temporal resolution out to a two week forecast. In doing so, we demonstrated a variety of techniques for interpreting the ANN results in terms of the properties of Noah-LSM and using physical reasoning on multiple scales, using observations from the full spatiotemporal domain to motivate further investigation of particular regions and individual pixel time series. We also introduced several modifications to the neural network training process that had notable effects on the subsequent model characteristics. Ultimately, we developed an open source software framework that is general enough for similar ANN development and evaluation techniques to be applied to other similar problems. Below we have listed a few of the specific observations we have made about ANNs and their ability to emulate Noah-LSM.

\begin{itemize}
    \item{ANNs that predict the increment change in soil moisture content between time steps are more skillful than those which target the magnitude of the soil moisture state.}
    \item{The inclusion of a structured hidden state which is passed between prediction time steps in the LSTM ANN architecture makes them more effective at capturing temporal dynamics than FNN-style models that only propagate their previous output state.}
    \item{Normalizing soil moisture target values between individual soil textures' wilting and saturation point aligns the dynamic range of different textures, which led to more performant models.}
    \item{ANN emulators struggled to capture frozen soil dynamics, especially for sandy soil textures. This behavior is likely related to the exclusion of soil temperature information from model inputs, which are critically important within Noah-LSM for governing soil hydrology in such scenarios.}
    \item{ANN emulators also had difficulty capturing the strong seasonal variation in transpiration rate associated with the cropland vegetation type}
    \item{Loss function manipulations like biasing loss amounts by the magnitude of target changes in moisture content can improve model performance in extreme cases, but introduce a trade-off with performance under normal conditions.}
    \item{Other loss function manipulations like feature-wise normalization coefficients can emphasize or de-emphasize certain outputs.}
\end{itemize}

\section{Future Work}

We are interested in adapting this technique to the NLDAS-3 dataset, which is currently under development using the next-generation Noah-MP as a land surface model. ANN emulation may be able to lower the computational cost and simplify the software environment associated with executing the land surface model in a forecasting capacity, which opens the opportunity for executing large ensembles, and for users to easily manipulate the inputs to evaluate hypothetical scenarios on custom domains without the difficulty of running the full numerical model. Additionally, due to data storage constraints, the NLDAS-3 land model data will only be publicly available at a daily resolution; similar strategies to the ones here may be employed to develop an ANN that interpolates the hourly land surface data from the forcings, which may be provided to users so that they can retrieve hourly land surface states without the need to store 24 times the amount of model data on-disc.

In terms of the continued improvement of the ANNs' capabilities as emulators, the performance of data-driven models is closely related to the statistical properties of the training set, and is limited by the extent to which the predictors can explain the variance of the target variables. As such, further advancement of ANN modeling techniques in this domain are likely to involve adjusting the training data to supplement deficiencies in certain important but under-represented phenomena, and by providing additional information that can mitigate uncertainty. For example, in order to encourage the ANNs to realize the seasonality of cropland vegetation, or to better capture the runoff characteristics of the sparse urban surface type, the training pipeline can be modified to select those sequences more often, or to weight their results more heavily within the loss function. Furthermore, since we are aware that necessary information related to soil temperature is missing from the frozen hydrology estimates, additional ANNs can be developed that emulate thermodynamic processes, and which can be coupled and used to constrain the hydrology emulator. In this manner, a suite of ANN emulators targeting specific physical variables (for example, also including heat fluxes and snow water equivalent) could mutually constitute a uniquely explainable framework for data-driven land surface modeling. In the future, such a system could assimilate physical observations, and may even be able to bootstrap emulation of actual in-situ data rather than just numerical model outputs.

Finally, there are a variety of improvements to the evaluation strategies used here that we would like to implement in the future. For example, the physicality of ANN performance could be characterized with respect to the magnitude of vegetation stress, the infiltration rate of particular soil textures, the actual fraction of frozen soil, etc. We would also like to add a evaluation method that builds a spatiotemporal map of occurrences of ANN behaviors, for example `absolute error over .2 RSM associated with precipitation events greater than 15 $kg\,m^{-2}\,h^{-1}$' or `dry bias with dewpoint depression less than 2 $K$'. Such a system would be incredibly useful for interpreting the circumstances contributing bulk results like those in the histograms presented here, and for identifying special cases of model behavior that can be used for over-sampling training data or identifying case studies. The flexibility of data-driven modeling makes it a promising and valuable technology for supplementing process-based knowledge and modeling approaches, but it is critical to recognize that the subtleties and vicissitudes of the Earth system cannot be sufficiently summarized by optimizing and analyzing the overall mean absolute error. As ANNs continue to evolve and the process of developing them becomes increasingly abstract, it will be critical for practitioners to maintain a physical understanding of the system they are emulating, and to evaluate their results accordingly.
