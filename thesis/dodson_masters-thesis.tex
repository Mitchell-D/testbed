% Below is your UAH LaTeX template. This is the main file that compiles your document. You will need to fill out the appropriate sections below.
\documentclass[oneside, 12pt]{book} % Document class


\usepackage[utf8]{inputenc} % Basic package for Latex
%\usepackage[draft]{graphicx} % To help include figures
\usepackage{listings}
\lstset{basicstyle=\ttfamily}
\usepackage{graphicx} % To help include figures
\usepackage{geometry} % To allow document margin changes
\usepackage{setspace} % To allow custom spacing
\usepackage{indentfirst} % To allow indentations in the first paragraph.
\usepackage{amsmath} % For formatting equations properly
\usepackage{natbib} % For formatting the bibliography
\usepackage{chngcntr} % To allow changing how figures, tables, and pages are counted.
\usepackage{appendix} % To help format the appendix
\usepackage[font=small,labelfont=bf]{caption} %This makes caption font small and makes the caption heading bold.
\usepackage{tabularx} %To allow formatting Tables.
\usepackage{array} %To allow creating arrays.
\usepackage{multirow, tabularx}
\usepackage{multicol}
\usepackage{physics}
\providecommand\phantomsection{} %This command allows you to make phantom sections that have no title but are still included in the Table of Contents such as the epigraph.
\usepackage[all]{nowidow} %This prevents widow/orphan lines
\usepackage{tikz}%This package helps to anchor the seal on the title page to the student name
\usetikzlibrary{tikzmark}%This is part of the tikz package


%Creating an Interlude Environment in order to NOT count or number a certain page such as the copyright page.
\newenvironment{interlude}{
  \clearpage
  \thispagestyle{empty}% we want this page to be empty (adjust to use a modified page style)
  \pagestyle{empty}% use the same style for subsequent pages in the unnumbered section
  }
  {\clearpage}

%The following package and lines can be used to format and create a list of symbols, equations, abbreviations, etc. The example actually used in this template is the \glossaries package, but you may also use this \nomencl package if desired.
\usepackage[intoc]{nomencl}
\makenomenclature
\renewcommand{\nomname}{List of Symbols}

% This is to ensure the page numbers are centered and at the bottom.
\usepackage{fancyhdr}
\pagestyle{fancy}
\fancyhf{}
\renewcommand{\headrulewidth}{0pt}
\cfoot{\thepage}

%This defines a new page style for just the title page.
\fancypagestyle{logopage}{\fancyhf{}\renewcommand{\headrulewidth}{0pt}\fancyfoot[C]{\includegraphics[scale=0.3]{Figures/Color Grad Banner.png}}}

%this causes equations to be counted according to their chapter location
\counterwithin{equation}{chapter} %this causes equations to be counted according to their chapter location
\counterwithin{figure}{chapter}




%This formats the table of contents the way Dr. Hakkila wants it
\usepackage[titles]{tocloft}


%Removes the Duplicate Chapter
\setcounter{secnumdepth}{4}% Show down to subsubsection

\setlength{\cftchapindent}{-20pt}% Just some value...

\usepackage{xpatch}

\makeatletter
\xpatchcmd{\@chapter}{\addcontentsline{toc}{chapter}{\protect\numberline{\thechapter}#1}}{%
                      \addcontentsline{toc}{chapter}{\protect\numberline{}#1}}{\typeout{Success}}{\typeout{Failed!}}
\makeatother



%This formats the Table of contents title
\renewcommand{\contentsname}{\hspace*{\fill}\bfseries\large Table of Contents\hspace*{\fill}}


\renewcommand{\cftbeforetoctitleskip}{-0.25in}

\renewcommand\cftchapdotsep{\cftdotsep}
\renewcommand\cftchapleader{\cftdotfill{\cftchapdotsep}}
%\renewcommand{\cftchappresnum}{CHAPTER } % put this before the number
%\addtolength{\cftchapnumwidth}{6em} % extra space for number and pre-name
%\renewcommand{\cftchapaftersnum}{. } % put period after chapter number and space


%This puts all the chapter headings into all-caps
%\renewcommand{\chaptername}{CHAPTER}

%Hopefully this reformats the List of Figures Title
\renewcommand{\listfigurename}{\hspace*{\fill}\bfseries\large List of Figures\hspace*{\fill}}

%Reformat List of Tables, etc.
\renewcommand{\listtablename}{\hspace*{\fill}\bfseries\large List of Tables\hspace*{\fill}}


%This creates a List of Equations.
\newcommand{\listequationsname}{List of Equations}
\newlistof{equations}{equ}{\listequationsname}

\newcommand{\eqdesc}[1]{%
  \csname phantomsection\endcsname % if hyperref is loaded
  \addcontentsline{equ}{equations}{\protect\numberline{\theequation}#1}%
}

\setlength{\parindent}{36pt}


%The following formats the chapter titles and sections appropriately
\usepackage{titlesec}



\titleformat
{\chapter}
[hang]
{\normalfont\large\bfseries\filcenter}
{} %Label. This is blank so that the entire chapter heading and title are properly centered.
{0pt} %  Horizontal Space between label and title body.
{} %Before-Code
{}

\titleformat
{\section}
[hang]
{\normalfont\normalsize\bfseries}
{\thetitle} %Label.
{.5em} %  Horizontal Space between label and title body.
{} %Before-Code
{}

\titleformat
{\subsection}
[hang]
{\normalfont\normalsize\bfseries}
{\thetitle} %Label.
{.5em} %  Horizontal Space between label and title body.
{} %Before-Code
{}

\titlespacing{\subsection}{2em}{12pt}{12pt}

\titlespacing{\chapter}{0pt}{50pt}{12pt}

\titlespacing{\part}{0pt}{12pt}{12pt}


\titlespacing{name=\chapter, numberless}{0pt}{0pt}{12pt}


% I had used this code to combine the chapter heading and
%\titleformat{\chapter}[hang]
 % {\normalfont\large\bfseries\filcenter}{\chaptertitlename\ \thechapter:}{.3em}{}


%This names the Bibliography/References whatever you want it to be.
\renewcommand{\bibname}{References}

%The below package creates the optional List of Symbols and formats it correction..
\usepackage[acronyms, automake, toc, nopostdot]{glossaries}
\usepackage{glossary-longbooktabs}
\newcolumntype{P}[1]{>{\centering\arraybackslash}p{#1}}
\renewcommand*{\entryname}{Symbol}
\newglossarystyle{mystyle}{% define custom glossaries style for Abbreviation page, read relative manual before change
\setglossarystyle{long-booktabs}%
\renewenvironment{theglossary}%
{\begin{longtable}{@{}P{3cm}@{}p{\dimexpr\linewidth-3cm}@{}}}%
{\end{longtable}}%

\renewcommand*{\glossaryheader}{%
 \bfseries\large Symbol & \centering\bfseries\large Description \tabularnewline\endhead \endfoot}%
}

\renewcommand{\glossarypreamble}{\normalsize}
\makeglossaries
 %To clean up this document, the preamble that includes the packages is located in the preamble.tex file found in the FrontMatter folder.

%********************************************
%********************************************
% THE SECTION BELOW MUST BE FILLED OUT
%********************************************
%********************************************

%BASIC INFORMATION
\newcommand{\thesistitle}{Deep Learning Time Series Prediction Strategies for Efficiently Emulating Noah Land Surface Model Soil Moisture Dynamics}
\newcommand{\studentname}{Mitchell T. Dodson}
\newcommand{\degree}{Master of Science in Atmospheric Science}
\newcommand{\department}{Atmospheric and Earth Science}
\newcommand{\gradyear}{2025}% complete 4 digit year, e.g., 2022
\newcommand{\gradmonth}{May}% Spell out the month completely, e.g., December
%\newcommand{\jointuni}{Auburn University} %If this is a joint degree, remove the % sign at the beginning of this line and enter the entire name of the additional universities.

%***********************
%SPECIFY THESIS OR DISSERTATION
% Below, if you are earning a master's degree, remove the "%" on the line below that says \newcommand{thesis}. If you are earning a PhD or other doctorate degree, remove the "%" on the line below that says \newcommand{dissertation}.

\newcommand{\thesis}{FOR MASTER'S STUDENTS ONLY}
%\newcommand{\dissertation}{FOR DOCTORATE DEGREE STUDENTS ONLY}

%*******************************
%SPECIFY THE PROFESSORS WHO WILL APPROVE YOUR THESIS/DISSERTATION
%Professor information: Fill out only their first and last name WITH NO PREFIXES OR SUFFIXES. If a line is not applicable, simply add a % sign at the beginning of that line. If there is an applicable line that has a % sign at the beginning, remove this sign and fill in as needed.
\newcommand{\resadv}{Christopher Hain}
\newcommand{\comchair}{Sundar Christopher}
%\newcommand{\reschair}{Research Advisor/Committee Chair Name} %If your research advisor and committee chair are the same person, enter his/her name on this line.
\newcommand{\commema}{Sean Freeman}
%\newcommand{\commemb}{[2nd Committee Member Name]}
%\newcommand{\commemc}{[3rd Committee Member Name]}
%\newcommand{\commemd}{[4th Committee Member Name]}
%\newcommand{\commeme}{[5th Committee Member Name]}
%\newcommand{\commemf}{[6th Committee Member Name]}
%\newcommand{\commemg}{[7th Committee Member Name]}
%\newcommand{\commemh}{[8th Committee Member Name]}
%\newcommand{\commemi}{[9th Committee Member Name]}
\newcommand{\depchair}{Lawrence Carey}
\newcommand{\colldean}{[College Dean Name]}
\newcommand{\graddean}{Jon Hakkila}

%**************
%INDICATE IF YOU HAVE REGISTERED FOR A COPYRIGHT.
%**************
%\newcommand{\copyrightreg}{Yes} %If you registered for a copyright through Proquest, please remove the % sign from the beginning of this line starting with \newcommand{\copyrightreg}{Yes}. If you did not register for a copyright, no action is needed.

%********************************************
%********************************************
% End of Section to Fill Out.
%********************************************
%********************************************

%This creates hyperlinks for chapter, figure, and table titles in your pdf. It should be the last package before the document begins.
\usepackage[colorlinks=true,linkcolor=black,anchorcolor=black,citecolor=black,filecolor=black,menucolor=black,runcolor=black,urlcolor=black]{hyperref}

%*********************************************
%*********************************************
%Optional List of Symbols/Abbreviations
%*********************************************
%*********************************************

%A list of symbols/abbreviations is optional. If you do not want to include one, simply delete this section from your document. You may also delete this section and use a different package as there are multiple packages that can be used to make a List of Symbols, Abbreviations, Nomenclature, Etc. Below uses the \glossaries-extra package. This package is nice if you want to include multiple lists or sections. The following links provide useful information on how to use this package.
% https://mirrors.mit.edu/CTAN/macros/latex/contrib/glossaries/glossariesbegin.pdf
% https://mirrors.rit.edu/CTAN/macros/latex/contrib/glossaries/glossaries-user.pdf
% https://www.overleaf.com/learn/latex/Glossaries


% This uses the glossaries package. With this package, you can include multiple types of lists, track page numbers if desired, and define new lists. More information can be found at the following sites: https://mirrors.mit.edu/CTAN/macros/latex/contrib/glossaries/glossariesbegin.pdf and https://mirrors.rit.edu/CTAN/macros/latex/contrib/glossaries/glossaries-user.pdf https://www.overleaf.com/learn/latex/Glossaries


%*****************************************
%Define your list of glossary items below. Remember that the entries that you enter in this file will not automatically appear in the List of Symbols. You also have to reference the symbol in the body of your thesis by using the \gls command. 

%symbols
\newglossaryentry{deg}{name=$^\circ$, description={Degree}}
\newglossaryentry{grav}{name={1D}, description={Normal gravity environment}}
\newglossaryentry{wf}{name={\textit{f}}, description={Wear factor}}
\newglossaryentry{alp}{name={$\alpha$},description={Alpha}}
\newglossaryentry{theta}{name={$r_O$}, description={ecosystem respiration at reference temperature $T_a=0{^\circ}$C}}
\newglossaryentry{te}{name={$\tau_e$}, description={precision of the normal distribution of the likelihood}}
\newglossaryentry{q10}{name={$Q_{10}$}, description={multiplication factor to respiration with 10$^\circ$C increases in $T_a$}}
\newglossaryentry{phi}{name={$\phi$}, description={vapour pressure deficit response function}}
\newglossaryentry{del}{name=$\delta$, description={Transition coefficient constant for the design of linear-phase FIR filters which are used to take up space when testing the list of symbols}}

 %Go to this page to enter all the symbols you plan to use in the document.

% Scroll down in this file to after the List of Tables section in order to actually add your Lists to your document,.

%**********************************************
%**********************************************
% End of Optional List of Symbols/Abbreviations
%**********************************************
%**********************************************



%******************************************
%******************************************
% Document Begins Here
%******************************************
%******************************************

\begin{document}

\frontmatter % This command creates the front matter environment.

%**********************************
%**********************************
%Title Page
%**********************************
%**********************************
% Do not directly edit this page. Everything on the Title Page should auto-fill if you correctly filled out the information on the main.tex file.

\begin{titlepage}
\newgeometry{left=1.5in,bottom=.5in}
    \begin{center}
        \Large
        \singlespacing
        \textbf{\thesistitle}
        

\vspace{2cm}

        \large
        \textbf{\studentname}\\
        \vspace{1.5cm}
        \normalsize
        \ifdefined\thesis
        \textbf{A THESIS}
        \vspace{1.5cm}
        \else
        \ifdefined\dissertation
        \textbf{A DISSERTATION}
        \vspace{1.5cm}
        \else
        Please identify this document as either a thesis or dissertation on the main.tex file in the section at the top that must be filled out.
        \vspace{1.5cm}
        \fi
        \fi

        \textbf{Submitted in partial fulfillment of the requirements \\for the degree of \degree}\\  
        
\vspace{0.1cm}
  \textbf{in}\\
      \vspace{0.1cm}
        \textbf{The Department of \department}\\
        \vspace{0.1cm}
  \textbf{to}\\
\vspace{0.1cm}
\textbf{The Graduate School}\\
\vspace{0.1cm}
\textbf{of}\\
\vspace{0.1cm}
        \ifdefined\jointuni
        \textbf{The University of Alabama in Huntsville\\ and\\  \jointuni}
        \else
        \textbf{The University of Alabama in Huntsville}
    \fi

        
        \vspace{0.4cm}
        \textbf{\gradmonth\ \gradyear}
        


    \end{center}
    

\vfill

\textbf{Approved by:}


\vspace{.1cm}
\setstretch{1.25}

\ifdefined\reschair
\noindent
Dr. \reschair, Research Advisor/Committee Chair\\
\else
\noindent
Dr. \resadv, Research Advisor \\
Dr. \comchair, Committee Chair\\
\fi
\ifdefined\commemi
\noindent
Dr. \commema, Committee Member\\
Dr. \commemb, Committee Member\\
Dr. \commemc, Committee Member\\
Dr. \commemd, Committee Member\\
Dr. \commeme, Committee Member\\
Dr. \commemf, Committee Member\\
Dr. \commemg, Committee Member\\
Dr. \commemh, Committee Member\\
Dr. \commemi, Committee Member\\
Dr. \depchair, Department Chair\\
Dr. \colldean, College Dean\\
Dr. \graddean, Graduate Dean\\
\else
\ifdefined\commemh
\noindent
Dr. \commema, Committee Member\\
Dr. \commemb, Committee Member\\
Dr. \commemc, Committee Member\\
Dr. \commemd, Committee Member\\
Dr. \commeme, Committee Member\\
Dr. \commemf, Committee Member\\
Dr. \commemg, Committee Member\\
Dr. \commemh, Committee Member\\
Dr. \depchair, Department Chair\\
Dr. \colldean, College Dean\\
Dr. \graddean, Graduate Dean\\
\else
\ifdefined\commemg
\noindent
Dr. \commema, Committee Member\\
Dr. \commemb, Committee Member\\
Dr. \commemc, Committee Member\\
Dr. \commemd, Committee Member\\
Dr. \commeme, Committee Member\\
Dr. \commemf, Committee Member\\
Dr. \commemg, Committee Member\\
Dr. \depchair, Department Chair\\
Dr. \colldean, College Dean\\
Dr. \graddean, Graduate Dean\\
\else
\ifdefined\commemf
\noindent
Dr. \commema, Committee Member\\
Dr. \commemb, Committee Member\\
Dr. \commemc, Committee Member\\
Dr. \commemd, Committee Member\\
Dr. \commeme, Committee Member\\
Dr. \commemf, Committee Member\\
Dr. \depchair, Department Chair\\
Dr. \colldean, College Dean\\
Dr. \graddean, Graduate Dean\\
\else
\ifdefined\commeme
Dr. \commema, Committee Member\\
Dr. \commemb, Committee Member\\
Dr. \commemc, Committee Member\\
Dr. \commemd, Committee Member\\
Dr. \commeme, Committee Member\\
Dr. \depchair, Department Chair\\
Dr. \colldean, College Dean\\
Dr. \graddean, Graduate Dean\\
\else
\ifdefined\commemd
Dr. \commema, Committee Member\\
Dr. \commemb, Committee Member\\
Dr. \commemc, Committee Member\\
Dr. \commemd, Committee Member\\
Dr. \depchair, Department Chair\\
Dr. \colldean, College Dean\\
Dr. \graddean, Graduate Dean\\
\else
\ifdefined\commemc
Dr. \commema, Committee Member\\
Dr. \commemb, Committee Member\\
Dr. \commemc, Committee Member\\
Dr. \depchair, Department Chair\\
Dr. \colldean, College Dean\\
Dr. \graddean, Graduate Dean\\
\else
\ifdefined\commemb
Dr. \commema, Committee Member\\
Dr. \commemb, Committee Member\\
Dr. \depchair, Department Chair\\
Dr. \colldean, College Dean\\
Dr. \graddean, Graduate Dean\\
\else
\ifdefined\commema
Dr. \commema, Committee Member\\
Dr. \depchair, Department Chair\\
Dr. \colldean, College Dean\\
Dr. \graddean, Graduate Dean\\
\else
Dr. \depchair, Department Chair\\
Dr. \colldean, College Dean\\
Dr. \graddean, Graduate Dean\\
\fi
\fi
\fi
\fi
\fi
\fi
\fi
\fi
\fi
\end{titlepage}


\restoregeometry%Your title page should self-generate after filling in the required information above.
\newpage

%This sets the page margins. If you plan to bind and print your thesis, change the left boarder to 1.5 in.
\newgeometry{left=1.5in, right=1in, bottom=1in, top=1in}
\setcounter{page}{2}
%**********************************
%**********************************
%Abstract Page
%**********************************
%**********************************

%The below code inserts your Abstract Page. While much of this page fills in automatically, you must go to the AbstractPage.tex file located in the FrontMatter folder and insert the actual text of your abstract.

% The top of your abstract will fill out automatically once you fill in the required fields on the main.tex file. In this file, you will provide your abstract body. Type your abstract body at the bottom of this page directly below the \doublespacing command.

\chapter{Abstract}
     \begin{center}
        \large
        \singlespacing
        \textbf{\thesistitle}\\
        \vspace{0.5cm}
        \large
        \textbf{\studentname}\\
        \vspace{0.5cm}
        \normalsize
        \ifdefined\thesis
        \textbf{A thesis submitted in partial fulfillment of the requirements \\for the degree of \degree}\\
        \else
        \ifdefined\dissertation
        \textbf{A dissertation submitted in partial fulfillment of the requirements \\for the degree of \degree}\\
        \else
        \textbf{Please identify this document as either a thesis or dissertation on the main.tex in the section at the top that must be filled out.}\\
    \fi
    \fi
        \vspace{1cm}
        \textbf{\department}

        \vspace{0.25cm}

        \ifdefined\jointuni
        \textbf{The University of Alabama in Huntsville and  \jointuni}
        \else
        \textbf{The University of Alabama in Huntsville}
    \fi


        \vspace{0.1cm}
        \textbf{\gradmonth\ \gradyear}



    \end{center}
\vspace{0.1cm}

%****************************************************
%Enter the body of your abstract below. Remember there is a 150 word limit!
%****************************************************
\doublespacing

This work examines the ability of deep learning time series generative models to accurately and efficiently emulate the hourly temporal dynamics of the Noah Land Surface Model (Noah-LSM) out to a 2 week forecast horizon, given atmospheric forcings and static parameterization provided by the second phase North American Land Data Assimilation System (NLDAS-2) framework. Results from multiple neural network architectures are compared alongside variations in prediction target, loss function characteristics, and model properties. The most performant model types are subsequently evaluated with respect to forecast distance, daily and annual seasonality, and against a variety of regional scenarios, including several extreme event case studies. Ultimately, we present a software system for developing and testing neural networks that use time-varying and static data to estimate temporal dynamics, with the goal of providing a foundation for similar data-driven modeling techniques to be implemented within the upcoming third phase of the NLDAS data record.

\clearpage



%**********************************
%**********************************
%Copyright Page
%**********************************
%**********************************

% The following code inserts your COPYRIGHT page. If you have registered for a copyright through Proquest, you should have removed the % sign from the \newcommand{copyrightreg} at the end of the section to Fill Out. If you have not registered for a copyright, no action is needed.
\ifdefined\copyrightreg
\doublespacing
\begin{center}

    \vspace*{\fill}

\copyright \\
\studentname \\
All Rights Reserved
\end{center}
\else
\newpage
\
\newpage
\fi



%**********************************
%**********************************
%Acknowledgements
%**********************************
%**********************************

% Your acknowledgements are included here. Similar to the abstract, you must open the Acknowledgements.tex file located in the FrontMatter folder to type your acknowledgements.
%% Type your Acknowledgements below. Delete all the text after the double-spacing command.
\chapter{Acknowledgements}
\doublespacing

I'd like to extend my gratitude to my technical advisor, Dr. Chris Hain, for his guidance and dedication toward me throughout the process of developing this work. This has been a long process with plenty of setbacks and diversions, and I am thankful for the patience, gentle correction, and freedom he and the rest of my committee has afforded me. My appreciation also goes to Dr. Sundar Christopher for his kindness in providing both scientific and administrative direction, and to Dr. Sean Freeman for engaging with interest and providing insight on the technical details of my project. Thanks also to Ryan Wade for being a protean beacon of knowledge and experience, and for offering me a wealth of advice and many opportunities, and to Paul Meyer for his close mentorship and many hours of conversation during my undergrad years.

Naturally, this wouldn't be possible without the help of a community of people around me. It has been a pleasure to forge strong bonds with my fellow graduate students as we overcome numerous challenges together. I also wouldn't have had the fortitude to finish this effort if it weren't for weekend escapes into caves and over mountains with my dear friends and my partner Shae, and the mental respite they provide. The entire institution of the National Space Science and Technology Center -- the SPoRT team in particular -- also has my appreciation and esteem; it is an honor and a source of constant motivation to work alongside such a diverse and passionate group of experts.

Finally, my greatest thanks goes to my family, who have continued to support and believe in me, even when they aren't sure where all my time is going. Without their confidence, love, and prayers, this process could not have even begun.

% Type your Acknowledgements below. Delete all the text after the double-spacing command.
\chapter{Acknowledgements}
\doublespacing

I'd like to extend my gratitude to my technical advisor, Dr. Chris Hain, for his guidance and dedication toward me throughout the process of developing this work. This has been a long process with plenty of setbacks and diversions, and I am thankful for the patience, gentle correction, and freedom he and the rest of my committee has afforded me. My appreciation also goes to Dr. Sundar Christopher for his kindness in providing both scientific and administrative direction, and to Dr. Sean Freeman for engaging with interest and providing insight on the technical details of my project. Thanks also to Ryan Wade for being a protean beacon of knowledge and experience, and for offering me a wealth of advice and many opportunities, and to Paul Meyer for his close mentorship and many hours of conversation during my undergrad years.

Naturally, this wouldn't be possible without the help of a community of people around me. It has been a pleasure to forge strong bonds with my fellow graduate students as we overcome numerous challenges together. I also wouldn't have had the fortitude to finish this effort if it weren't for weekend escapes into caves and over mountains with my dear friends and my partner Shae, and the mental respite they provide. The entire institution of the National Space Science and Technology Center -- the SPoRT team in particular -- also has my appreciation and esteem; it is an honor and a source of constant motivation to work alongside such a diverse and passionate group of experts.

Finally, my greatest thanks goes to my family, who have continued to support and believe in me, even when they aren't sure where all my time is going. Without their confidence, love, and prayers, this process could not have even begun.


%The code below formats your table of contents, list of figures, list of tables, and list of symbols. If your document does not contain any figures and/or tables, simply delete that section. The list of symbols is optional. Again, delete that section if you do not want to include it in your document.

%*************************
%Table of Contents Section
%*************************
\newgeometry{left=1.75in}%For some reason, I have to set only the Table of Contents to a left margin of 1.75 so that everything lines up correctly.
{\renewcommand\uppercase[1]{#1} % This creates an environment to NOT put titles in all-caps
\singlespacing
\setlength{\cftparskip}{1\baselineskip}% This single-spaces within entries and double-spaces between them.
\tableofcontents %Command to create the table of contents
\addcontentsline{toc}{chapter}{Table of Contents} %Changes the name from Contents to Table of Contents
\newpage %Creates a page break before the next section.

%*************************
%List of Figures Section
%*************************
\newgeometry{left=1.5in}
\cleardoublepage\phantomsection\addcontentsline{toc}{chapter}{List of Figures} %Adds the List of figures to the Table of contents.
\singlespacing
\setlength{\cftparskip}{.5\baselineskip} %This allows single space within entries and double space between them.
\listoffigures %This command creates the list of figures.
\newpage %Creates a page break before the next section.

%*************************
%List of Tables Section
%*************************
\newgeometry{left=1.5in}
\cleardoublepage\phantomsection\addcontentsline{toc}{chapter}{List of Tables} %Adds the List of figures to the Table of contents. The \clear doublepage and \phantomsection make the links work properly.
\singlespacing
\setlength{\cftparskip}{.5\baselineskip} %This allows single space within entries and double space between them.
\listoftables %This command creates the list of tables.
\newpage
}


%**************************
%List of Symbols Section
%**************************
\singlespacing
\renewcommand*{\arraystretch}{2}
\printglossary[title=\centering List of Symbols, toctitle=List of Symbols,style=mystyle,nonumberlist]



%Include your epigraph here if you have one by removing the % sign on the lines of code below and then typing in the required information on the epigraph page. Go to the epigraphOptional.tex file found in the FrontMatter folder.

%\clearpage \phantomsection \addcontentsline{toc}{chapter}{Epigraph} % This page is optional. If you plan to include it, simply insert your quote and the author in the appropriate locations below. 
\newgeometry{top=2in}
\begin{center}
    \textit{My Quote}
\end{center}
\begin{flushright}
- [Author Name]
\end{flushright}
\restoregeometry

%************************
%Body of your Thesis Begins
%************************
\mainmatter

\newgeometry{left=1.5in}
\doublespacing
%The contents of your chapters are located in separate chapter.tex files. This template only contains 3 chapter files (ch1, ch2, and chLast). To edit these files, open the corresponding chapter.tex files. To create new chapters, make a new .tex file for each chapter and then insert them into your document below with the \include{name of your chapter .tex file} command.

%
\chapter{Chapter 1. Introduction}%Be sure to include Chapter 1. before you write the name of your chapter. Name all remaining chapters in the same manner.

Accurate characterization of the distribution of water content within the soil column by land surface models is critical for governing land-atmosphere interaction in numerical weather prediction (NWP) \citep{brocca_spatial-temporal_2010} \citep{koster_contribution_2010}, operational decision making preceding and during drought and flood events \citep{otkin_assessing_2016}, and for downstream datasets aiding assessment of vegetation health, crop yield prediction, and fire risk characterization \citep{case_role_2023}. In order to address these needs, the Noah-LSM was developed to serve as the land surface component coupled to NWP models including the Weather Research and Forecasting Model (WRF), the Global Forecast System (GFS) \citep{jin_sensitivity_2010}\citep{mitchell_ncep_2005}, and climate models including the NCEP Climate Forecast System (CFSv2) \citep{saha_ncep_2014}. Noah-LSM also aids National Weather Service forecasts and US Drought Monitor designations within decision support frameworks like the Short-Term Research, Prediction, and Transition high-resolution implementation of the Land Information System (SPoRT-LIS) \citep{case_nasa_2022}\citep{case_assessment_2014}, and facilitates research and derived product development by providing soil states for NASA Land Data Assimilation System (LDAS) datasets \citep{ek_implementation_2003}.

By applying observational and reanalysis data to Noah and other land surface models, NLDAS has provided the community with consistent and quality-controlled multi-model land surface states and associated forcings in a near real-time capacity since 1999 \citep{cosgrove_real-time_2003}, with phase 2 of the project also contributing a retrospective climatology extending back to 1979. The first and second generation data products are calculated on a 1/8 degree geodetic grid spanning land-dominated points in the conterminous United States (CONUS) from 25$^\circ$ to 53$^\circ$ North latitude and 125$^\circ$-67$^\circ$ West longitude, and are released at an hourly frequency  \citep{mitchell_multi-institution_2004}\citep{xia_continental-scale_2012}. The third phase of the data assimilation system is currently under development, and aims to implement a wealth of upgrades including new data assimilation techniques and physical parameterizations, an increase in the spatial resolution to 1km$^2$, and the expansion of the domain to the full North American continent. As a consequence, the total number of valid land grid cells will increase dramatically from 76,088 in the first two phases to 27,245,580 with NLDAS-3 data products. In addition to the larger domain and updated physical processes used to develop the forcings and land surface states, the NLDAS-3 data suite will feature a variety of derived products. These products are anticipated include gridded climatological anomaly and segmented percentile data, stream routing and discharge estimates, and ensemble mean and spread information using forecast forcings \citep{kumar_north_2024}.

As the domain size and sophistication of data assimilation systems and land surface models like NLDAS and Noah-LSM continues to grow, a niche develops for methods that can generate reasonable estimates of the dynamics of numerical models which require less compute time, simplify the runtime environment of the program, and which can be fitted to observational data and then generalized to broader domains without accruing significant additional complexity to the parameterization scheme.  Data-driven modeling techniques like deep learning with artificial neural networks (ANNs) are addressing this need by introducing the ability to approximate the highly nonlinear and conditional relationships between arbitrary predictor and target datasets. This flexibility is accomplished by learning a sequence of transformations which are encoded as a composition of alternating high-dimensional matrix operations and element-wise nonlinear functions, and which serve as a mapping from the vector of predictors to a corresponding target vector \citep{hornik_multilayer_1989}.

In the context of time series physical modeling, ANNs enable the development of a statistically optimal approximation of the relationship between past states, simultaneous covariate data variables, and unknown current or future states. This general principle has a wealth of use cases. Previous literature shows that ANNs are computationally efficient and reasonably accurate for modeling dynamical systems like Lorenz'95 by formulating the problem as a discrete-time estimator of an ordinary differential equation which isn't explicitly known by the model \citep{fablet_bilinear_2018}. ANNs can also be structured to have useful properties like the ability to estimate the jacobian of the transfer mapping between inputs and future states, even if the system being emulated isn't differentiable \citep{nonnenmacher_deep_2021}. The same strategy may be applied to forecasting the evolution datasets like ECMWF Reanalysis v5 (ERA5) in a local or global domain, however significant challenges emerge as \citep{dueben_challenges_2018} identify. As they describe, ANNs cannot be constrained by default to conserve quantities like energy and water, and unlike numerical models their handling of the underlying physical processes as a ``black-box'' mean that identifying sources of error within the model is difficult and often speculative. Furthermore, Earth system data tend to be highly regionally variable (ex. vegetation types), exhibit nonlinear autocorrelation between multiple variables (ex. temperature, dewpoint, and cloud cover), and are subject to rare but influential outliers (ex. snow and extreme precipitation). As such, although ANNs are adept at handling very nonlinear and conditional problem types, achieving the best performance and interpretability requires the utilization of application-specific knowledge when constructing and evaluating deep learning models.

Within the field of hydrologic modeling, most of the recent literature applying deep learning methods has focused on rainfall-runoff problems, where models forecast the hydrograph of a stream given time-varying atmospheric and land surface states as well as static properties. Inputs are typically considered within a spatial boundary drawn from a watershed outlet where a streamflow station provides the prediction target by directly observing the discharge. To that end, \citep{kratzert_rainfallrunoff_2018} applies a particular ANN architecture called Long Short-Term Memory (LSTM) networks to modeling discharge from the CAMELS dataset \citep{addor_camels_2017}, which contains daily-resolution streamflow and meteorological forcings alongside parameters describing the topographic, land use, soil, and geologic properties of 671 catchments. They show that models trained on single basins often outperform models trained using data from multiple basins within a region, and that subsequent ``fine-tuning'' of a generalized regional model on individual basins slightly improves model efficiency in many cases. Later, \citep{kratzert_towards_2019} improves on LSTM model performance by modifying the training strategy to optimize an objective function similar to nash-sutcliffe efficiency, and by introducing a modification to the architecture that allows for static catchment parameters to be separately provided -- and their influence separately investigated -- from time-varying inputs. These experiments even out-performed several process-based models that were tuned specifically to the individual test basins. In spite of their black-box nature, \citep{lees_hydrological_2022} demonstrates that LSTMs used for daily-scale rainfall-runoff prediction maintain information correlated with physical properties of the catchment's hydrologic state including soil moisture and snow cover, which indicates that they preserve meaningfully interpretable data about their inputs. The general approach of employing LSTMs for discharge forecasting is already being utilized by stakeholders like the United States National Weather Service and River Forecast Center offices in an operational setting with the NASA SPoRT Streamflow-AI product, which uses near real-time Noah-LSM soil moisture estimates and outlooks as an input via the SPoRT-LIS data product \citep{white_nasa_2025}, \citep{case_nasa_2022}.

Relatively few publications have applied deep learning techniques to estimate soil dynamics over a consistently spatially gridded domain, akin to the outputs of process-based models like Noah-LSM.  In one instance, \citep{filipovic_regional_2022} applied LSTMs to global daily-scale ERA5 data in order to predict the 3-day evolution of moisture content in an intermediate-depth soil layer. This is conceptually similar to emulating Noah-LSM using NLDAS forcings because ERA5 determines its soil moisture states using the ECMWF Scheme for Surface Exchanges over Land \citep{balsamo_revised_2009}. Additionally, \citep{o_global_2021} used an LSTM to assist in generalizing in-situ observations at 3 soil depth levels to a regional grid, also using daily ERA5 forcings data as an input, and adjusting predictions to match the pixel-wise gaussian parameters of the ERA5 soil moisture analysis. Both of these approaches use long lead times of 60 days or 1 year, respectively, and make predictions at only a few forecast horizons per execution of the model (3 days and 1 day, respectively).

This work seeks to apply a similar strategy of data-driven modeling for hourly-scale emulation of Noah-LSM over the full NLDAS-2 grid domain, with the goal of generating accurate and computationally reasonable forecasts out to a two-week horizon at three depth levels. We will construct a few distinct neural network types suited to this problem structure, compare their results through a variety of bulk statistics and case studies using physical reasoning, discuss lessons learned regarding training methodology, and present a general free and open-source framework for developing time series dynamical estimators using deep learning for gridded physical datasets.

\newpage

\section{History of NLDAS and Noah-LSM}

Chapter titles should begin with the word chapter and the appropriate number followed by a period. After typing the chapter heading, then type the chapter title. This template automatically formats your chapter titles. Just do not forget to include the chapter heading when you type the chapter name.

All paragraphs throughout your thesis should begin with an ½ inch indentation. It should be double-spaced throughout. Since this is a formal document, do not use contractions. Remember that paragraphs should consist of at least two sentences. Figure 1.1 lists 11 common grammar mistakes. Please avoid these!

\begin{figure}[ht]
    \centering
    \includegraphics[width=\textwidth]{Figures/figure 1.1.jpg}
    \caption[11 Most Common Grammar Mistakes Employees Make: I'm purposely making this longer to extend to two lines.]{11 Most Common Grammar Mistakes Employees Make. When labeling your figures, single-space if captions extend to two lines}
    \label{fig 1.1}
\end{figure}

\section{Deep Learning for Time Series Modeling}

If your document includes many symbols or acronyms, you may include a List of Symbols, Abbreviations, \textit{etc}. If you want a symbol/abbreviation included in the List of Symbols, be sure to create an entry for it first on the List of Symbols Glossaries.tex file. Once it is created, then you can insert it with a glossaries command. For example, the current temperature outside is 100\glspl{deg}.

You can capitalize your symbols or make them plural by using different commands included with the glossaries package. However, only those symbols that are actually referenced in the body of your thesis will be present in the List of Symbols. Below are a few more symbol examples.

\gls{grav}

\gls{wf}

\gls{alp}

\gls{theta}

\gls{te}

\gls{q10}




%\chapter{Chapter 2. Adding New Chapter, Creating Sections or Subsections, and Formatting Equations}

\section{Adding New Chapters}
You may use the chapter.tex files already contained in this template for chapters 1, 2, and your concluding chapter. However, any additional chapters will need to be created in a separate .tex file and then inserted to your main.tex file with the include command. This template includes examples of how to properly format content, but feel free to delete all the content in these chapter.tex files in order to add your own content. 

\section{Creating Sections or Subsections}
When adding sections or subsections, simply use the section or subsection command. LaTeX will format the title of the section and/or subsection correctly automatically. Also, if you use Overleaf as the editor, it automatically has a spelling check with is very convenient. 
\subsection{Formatting Equations}
Latex automatically assigns equations numbers based on their location in the document. However, if you want to reference specific equations throughout your work, you will need to manually provide an internal label. Below is an example equation with a created label followed with a reference to this equation.

\begin{equation}
  \label{example}
  \begin{split}
   \nabla \cdot \nabla \psi &= \frac{\partial^2 \psi}{\partial x^2} + \frac{\partial^2 \psi}{\partial y^2} + \frac{\partial^2 \psi}{\partial z^2} \\
   &= \frac{1}{r^2 \sin\theta} \left[ \sin\theta \left( r^2 \frac{\partial \psi}{\partial r} \right) + \frac{\partial}{\partial \theta} \left( \sin \theta  \frac{\partial \psi}{\partial r} \right) + \frac{1}{\sin \theta} \frac{\partial^2 \psi}{\partial \varphi^2}  \right] 
     \end{split}
\end{equation}
Equation \ref{example} will hopefully help you understand how to properly format and reference equations in your document.

\subsection{Citations}
When you make your citations, you will need to first add them to the ref.bib file. Then, use the citation command followed by the name of the citation.\cite{Example:1} LaTeX allows you to control the style of your citations.\cite{Example:2} On the main.tex file, set your bibliography style to the one you prefer. 




%include{Chapters/Ch3}

\chapter{Chapter 1. Introduction}%Be sure to include Chapter 1. before you write the name of your chapter. Name all remaining chapters in the same manner.

Accurate characterization of the distribution of water content within the soil column by land surface models is critical for governing land-atmosphere interaction in numerical weather prediction (NWP) \citep{brocca_spatial-temporal_2010} \citep{koster_contribution_2010}, operational decision making preceding and during drought and flood events \citep{otkin_assessing_2016}, and for downstream datasets aiding assessment of vegetation health, crop yield prediction, and fire risk characterization \citep{case_role_2023}. In order to address these needs, the Noah-LSM was developed to serve as the land surface component coupled to NWP models including the Weather Research and Forecasting Model (WRF), the Global Forecast System (GFS) \citep{jin_sensitivity_2010}\citep{mitchell_ncep_2005}, and climate models including the NCEP Climate Forecast System (CFSv2) \citep{saha_ncep_2014}. Noah-LSM also aids National Weather Service forecasts and US Drought Monitor designations within decision support frameworks like the Short-Term Research, Prediction, and Transition high-resolution implementation of the Land Information System (SPoRT-LIS) \citep{case_nasa_2022}\citep{case_assessment_2014}, and facilitates research and derived product development by providing soil states for NASA Land Data Assimilation System (LDAS) datasets \citep{ek_implementation_2003}.

By applying observational and reanalysis data to Noah and other land surface models, NLDAS has provided the community with consistent and quality-controlled multi-model land surface states and associated forcings in a near real-time capacity since 1999 \citep{cosgrove_real-time_2003}, with phase 2 of the project also contributing a retrospective climatology extending back to 1979. The first and second generation data products are calculated on a 1/8 degree geodetic grid spanning land-dominated points in the conterminous United States (CONUS) from 25$^\circ$ to 53$^\circ$ North latitude and 125$^\circ$-67$^\circ$ West longitude, and are released at an hourly frequency  \citep{mitchell_multi-institution_2004}\citep{xia_continental-scale_2012}. The third phase of the data assimilation system is currently under development, and aims to implement a wealth of upgrades including new data assimilation techniques and physical parameterizations, an increase in the spatial resolution to 1km$^2$, and the expansion of the domain to the full North American continent. As a consequence, the total number of valid land grid cells will increase dramatically from 76,088 in the first two phases to 27,245,580 with NLDAS-3 data products. In addition to the larger domain and updated physical processes used to develop the forcings and land surface states, the NLDAS-3 data suite will feature a variety of derived products. These products are anticipated include gridded climatological anomaly and segmented percentile data, stream routing and discharge estimates, and ensemble mean and spread information using forecast forcings \citep{kumar_north_2024}.

As the domain size and sophistication of data assimilation systems and land surface models like NLDAS and Noah-LSM continues to grow, a niche develops for methods that can generate reasonable estimates of the dynamics of numerical models which require less compute time, simplify the runtime environment of the program, and which can be fitted to observational data and then generalized to broader domains without accruing significant additional complexity to the parameterization scheme.  Data-driven modeling techniques like deep learning with artificial neural networks (ANNs) are addressing this need by introducing the ability to approximate the highly nonlinear and conditional relationships between arbitrary predictor and target datasets. This flexibility is accomplished by learning a sequence of transformations which are encoded as a composition of alternating high-dimensional matrix operations and element-wise nonlinear functions, and which serve as a mapping from the vector of predictors to a corresponding target vector \citep{hornik_multilayer_1989}.

In the context of time series physical modeling, ANNs enable the development of a statistically optimal approximation of the relationship between past states, simultaneous covariate data variables, and unknown current or future states. This general principle has a wealth of use cases. Previous literature shows that ANNs are computationally efficient and reasonably accurate for modeling dynamical systems like Lorenz'95 by formulating the problem as a discrete-time estimator of an ordinary differential equation which isn't explicitly known by the model \citep{fablet_bilinear_2018}. ANNs can also be structured to have useful properties like the ability to estimate the jacobian of the transfer mapping between inputs and future states, even if the system being emulated isn't differentiable \citep{nonnenmacher_deep_2021}. The same strategy may be applied to forecasting the evolution datasets like ECMWF Reanalysis v5 (ERA5) in a local or global domain, however significant challenges emerge as \citep{dueben_challenges_2018} identify. As they describe, ANNs cannot be constrained by default to conserve quantities like energy and water, and unlike numerical models their handling of the underlying physical processes as a ``black-box'' mean that identifying sources of error within the model is difficult and often speculative. Furthermore, Earth system data tend to be highly regionally variable (ex. vegetation types), exhibit nonlinear autocorrelation between multiple variables (ex. temperature, dewpoint, and cloud cover), and are subject to rare but influential outliers (ex. snow and extreme precipitation). As such, although ANNs are adept at handling very nonlinear and conditional problem types, achieving the best performance and interpretability requires the utilization of application-specific knowledge when constructing and evaluating deep learning models.

Within the field of hydrologic modeling, most of the recent literature applying deep learning methods has focused on rainfall-runoff problems, where models forecast the hydrograph of a stream given time-varying atmospheric and land surface states as well as static properties. Inputs are typically considered within a spatial boundary drawn from a watershed outlet where a streamflow station provides the prediction target by directly observing the discharge. To that end, \citep{kratzert_rainfallrunoff_2018} applies a particular ANN architecture called Long Short-Term Memory (LSTM) networks to modeling discharge from the CAMELS dataset \citep{addor_camels_2017}, which contains daily-resolution streamflow and meteorological forcings alongside parameters describing the topographic, land use, soil, and geologic properties of 671 catchments. They show that models trained on single basins often outperform models trained using data from multiple basins within a region, and that subsequent ``fine-tuning'' of a generalized regional model on individual basins slightly improves model efficiency in many cases. Later, \citep{kratzert_towards_2019} improves on LSTM model performance by modifying the training strategy to optimize an objective function similar to nash-sutcliffe efficiency, and by introducing a modification to the architecture that allows for static catchment parameters to be separately provided -- and their influence separately investigated -- from time-varying inputs. These experiments even out-performed several process-based models that were tuned specifically to the individual test basins. In spite of their black-box nature, \citep{lees_hydrological_2022} demonstrates that LSTMs used for daily-scale rainfall-runoff prediction maintain information correlated with physical properties of the catchment's hydrologic state including soil moisture and snow cover, which indicates that they preserve meaningfully interpretable data about their inputs. The general approach of employing LSTMs for discharge forecasting is already being utilized by stakeholders like the United States National Weather Service and River Forecast Center offices in an operational setting with the NASA SPoRT Streamflow-AI product, which uses near real-time Noah-LSM soil moisture estimates and outlooks as an input via the SPoRT-LIS data product \citep{white_nasa_2025}, \citep{case_nasa_2022}.

Relatively few publications have applied deep learning techniques to estimate soil dynamics over a consistently spatially gridded domain, akin to the outputs of process-based models like Noah-LSM.  In one instance, \citep{filipovic_regional_2022} applied LSTMs to global daily-scale ERA5 data in order to predict the 3-day evolution of moisture content in an intermediate-depth soil layer. This is conceptually similar to emulating Noah-LSM using NLDAS forcings because ERA5 determines its soil moisture states using the ECMWF Scheme for Surface Exchanges over Land \citep{balsamo_revised_2009}. Additionally, \citep{o_global_2021} used an LSTM to assist in generalizing in-situ observations at 3 soil depth levels to a regional grid, also using daily ERA5 forcings data as an input, and adjusting predictions to match the pixel-wise gaussian parameters of the ERA5 soil moisture analysis. Both of these approaches use long lead times of 60 days or 1 year, respectively, and make predictions at only a few forecast horizons per execution of the model (3 days and 1 day, respectively).

This work seeks to apply a similar strategy of data-driven modeling for hourly-scale emulation of Noah-LSM over the full NLDAS-2 grid domain, with the goal of generating accurate and computationally reasonable forecasts out to a two-week horizon at three depth levels. We will construct a few distinct neural network types suited to this problem structure, compare their results through a variety of bulk statistics and case studies using physical reasoning, discuss lessons learned regarding training methodology, and present a general free and open-source framework for developing time series dynamical estimators using deep learning for gridded physical datasets.

\newpage

\section{History of NLDAS and Noah-LSM}

Chapter titles should begin with the word chapter and the appropriate number followed by a period. After typing the chapter heading, then type the chapter title. This template automatically formats your chapter titles. Just do not forget to include the chapter heading when you type the chapter name.

All paragraphs throughout your thesis should begin with an ½ inch indentation. It should be double-spaced throughout. Since this is a formal document, do not use contractions. Remember that paragraphs should consist of at least two sentences. Figure 1.1 lists 11 common grammar mistakes. Please avoid these!

\begin{figure}[ht]
    \centering
    \includegraphics[width=\textwidth]{Figures/figure 1.1.jpg}
    \caption[11 Most Common Grammar Mistakes Employees Make: I'm purposely making this longer to extend to two lines.]{11 Most Common Grammar Mistakes Employees Make. When labeling your figures, single-space if captions extend to two lines}
    \label{fig 1.1}
\end{figure}

\section{Deep Learning for Time Series Modeling}

If your document includes many symbols or acronyms, you may include a List of Symbols, Abbreviations, \textit{etc}. If you want a symbol/abbreviation included in the List of Symbols, be sure to create an entry for it first on the List of Symbols Glossaries.tex file. Once it is created, then you can insert it with a glossaries command. For example, the current temperature outside is 100\glspl{deg}.

You can capitalize your symbols or make them plural by using different commands included with the glossaries package. However, only those symbols that are actually referenced in the body of your thesis will be present in the List of Symbols. Below are a few more symbol examples.

\gls{grav}

\gls{wf}

\gls{alp}

\gls{theta}

\gls{te}

\gls{q10}




\chapter{Chapter 2. Adding New Chapter, Creating Sections or Subsections, and Formatting Equations}

\section{Adding New Chapters}
You may use the chapter.tex files already contained in this template for chapters 1, 2, and your concluding chapter. However, any additional chapters will need to be created in a separate .tex file and then inserted to your main.tex file with the include command. This template includes examples of how to properly format content, but feel free to delete all the content in these chapter.tex files in order to add your own content. 

\section{Creating Sections or Subsections}
When adding sections or subsections, simply use the section or subsection command. LaTeX will format the title of the section and/or subsection correctly automatically. Also, if you use Overleaf as the editor, it automatically has a spelling check with is very convenient. 
\subsection{Formatting Equations}
Latex automatically assigns equations numbers based on their location in the document. However, if you want to reference specific equations throughout your work, you will need to manually provide an internal label. Below is an example equation with a created label followed with a reference to this equation.

\begin{equation}
  \label{example}
  \begin{split}
   \nabla \cdot \nabla \psi &= \frac{\partial^2 \psi}{\partial x^2} + \frac{\partial^2 \psi}{\partial y^2} + \frac{\partial^2 \psi}{\partial z^2} \\
   &= \frac{1}{r^2 \sin\theta} \left[ \sin\theta \left( r^2 \frac{\partial \psi}{\partial r} \right) + \frac{\partial}{\partial \theta} \left( \sin \theta  \frac{\partial \psi}{\partial r} \right) + \frac{1}{\sin \theta} \frac{\partial^2 \psi}{\partial \varphi^2}  \right] 
     \end{split}
\end{equation}
Equation \ref{example} will hopefully help you understand how to properly format and reference equations in your document.

\subsection{Citations}
When you make your citations, you will need to first add them to the ref.bib file. Then, use the citation command followed by the name of the citation.\cite{Example:1} LaTeX allows you to control the style of your citations.\cite{Example:2} On the main.tex file, set your bibliography style to the one you prefer. 




\chapter{Chapter 3. Data and Methodology}

\begin{table}[ht]

\caption[Frequencies for equal-tempered scale, $A_4=440$]{Frequencies for equal-tempered scale, $A_4=440$ Hz. This table shows only the first five notes of a chromatic scale starting on $C_0$} %Just provide the title of the table in the square brackets. Then, in the next set of brackets, provide the entire caption (including the title again). By doing this, only the title of the table will be on the List of Tables instead of the entire caption. The first sentence of the caption can be the title.

\begin{center}
\resizebox{\columnwidth}{!}{
\begin{tabular}{|c | c | c |}
 \hline
\textbf{Note} & \textbf{Frequency (Hz)} & \textbf{Wavelength} \\ [0.5ex]
 \hline
 $C_0$ & 16.35 & 2109.89 \\
 \hline
 $C^{\#}_0/D^b_0$ & 17.32 & 1991.47 \\
 \hline
 $D_0$ & 18.35 & 1879.69 \\
 \hline
 $D^{\#}_0/E^b_0$ & 19.45 & 1774.20 \\
 \hline
 $E_0$ & 20.60 & 1674.62 \\ [1ex]
 \hline
\end{tabular}
}
\end{center}


\end{table}



%************************
%Back Matter of your Thesis Begins
%************************
\backmatter

%***********************
%References
%***********************
\addcontentsline{toc}{chapter}{References} %This adds the Bibliography/References to your table of contents.
\bibliographystyle{plain} %This selects your bibliography style. There are many possible bibliography styles you can choose. The following site explains the options. https://www.overleaf.com/learn/latex/Bibtex_bibliography_styles


\begingroup %Begins an editable environment to set proper spacing for the bibliography/references page.
\setlength{\bibsep}{12pt} %This provides 10 pts between reference entries.
\setstretch{1} %This specifies single-spacing within entries.
\bibliography{ref} %This inserts your References. You must individually add all references to the ref.bib file. Then, only those references you actually cite in the body of your text will be included.
\endgroup

%***********************
%Appendix Section: Optional. If you do not want to include any appendices, simply delete the below commands that create the appendix environment and that input your appendix file(s).
%**********************
\appendix %This creates the appendix environment.

\chapter{Appendix A: An Example Appendix}%Be sure to include the Heading Appendix A: before you type the name of the Appendix.

\renewcommand{\thechapter}{A} %If you add another appendix, copy and paste this line, but update it to B instead of A.

Appendices should appear at the very end of your thesis. Make sure to label each Appendix with a letter starting with "A". Any tables and/or figures located in the appendix should be labeled accordingly. For example, below is figure A.1 because it is the first figure that appears in Appendix A. 


\begin{figure}[ht]
    \centering
    \includegraphics[width=\textwidth]{Figures/Figure A.1.png}
    \caption[Colleges and Universities in Alabama]{Colleges and Universities in Alabama}
    \label{fig a.1}
\end{figure}


 %This inserts your Appendix file(s). To edit this page, open the Appendix A.tex file. You will need to create a new .tex file for each appendix you want to include.

\end{document}
